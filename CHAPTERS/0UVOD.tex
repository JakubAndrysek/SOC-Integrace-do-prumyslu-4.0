\chapter*{Úvod}
\addcontentsline{toc}{chapter}{Úvod}

Cílem práce je navrhnout ucelený systém monitorující chod pletacích strojů ve firmě a~přizpůsobit ho co možná nejlépe potřebám firmy.
% \fxnote[author=JPA]{\textcolor{mygreen}{Jakých stojů? Co takhle přidat "střádacích" nebo něco v~tomto stylu?}}

S~nápadem vytvořit takovýto systém přišel můj děda, zakladatel firmy na výrobu ponožek.
Jeho snem vždy bylo mít takový systém, který by částečně zastal monotónní lidskou práci a~nahradil ji efektivní automatizací.

Můj systém jsem tedy navrhoval na míru pro rodinou firmu na pletení ponožek, ve které je okolo 25 pletacích strojů. 
Tento systém je schopen v~reálném čase zaznamenávat a~následně odesílat naměřená data ze strojů na server. 
Pro uživatele pak systém nabízí moderní webové stránky, kde si může naměřená data přehledně zobrazit a~analyzovat.

Podle pletacích strojů na kterých tento systém běží jsem projekt pojmenoval Pletačka IoT. 
Systém se skládá ze tří částí, senzorová část, která je připojená k~pletacímu stroji a~odesílá data.
Dále pak server, který veškerá data zpracovává a~zobrazuje je uživateli.
Poslední částí je podpůrný server, který se stará o~aktualizaci a~o~kontrolu správného chodu senzorů.\newline


Při~vytváření tohoto projektu jsem si dal za cíl
\begin{itemize}
    \item projekt s~otevřeným zdrojovým kódem
    \item cenová dostupnost
    \item jednoduché přidání senzorů
    \item přehledné uživatelské rozhraní
\end{itemize}

Pro systém jsem si stanovil tyto požadavky
\begin{itemize}
    \item Počítání upletených ponožek
    \item Zjišťování poruchovosti strojů
    \item Porovnání jednotlivých pracovních směn
    \item Monitorování průběhu výroby
\end{itemize}

\newpage
