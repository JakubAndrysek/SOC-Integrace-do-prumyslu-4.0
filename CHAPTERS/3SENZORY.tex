\chapter{Senzory}

Před začátkem navrhování senzorů bylo nutné si sepsat požadavky na senzory.




\section{1. verze - univerzální sensorika}

Při návrhu první verze senzoru jsem se držel těchto bodů
\begin{itemize}
    \item vstup ze 4 periferií
    \item senzor s barevným displayem
    \item vstupní napětí od 8 do 25V
    \item teplotní čidlo
    \item tři barevné diody
    \item čtyři uživatelská tlačítka
\end{itemize}


\subsection{Řídící deska}
Návrh desky jsem tvořil v aplikaci EAGLE od společnosti Autosk. 
Deska má rozměry 75 na 60 mm a v každém rohu má upevňovací díry. 
Kabely se do desky připojují pomocí ***5mm**** svorkovnice.
Na vstupu napájení je měnič napětí na 5V. 

Řídící procesor celé desky je modul ESP32 T-Display, který má 4MB paměti na program.
Tento čip také zajišťuje WiFi konektivitu s okolím a odesílání naměřených dat na server.

Pro bezpečný vstup z periferií se využívá optočlen, který předává signály do mikroprocesoru.
\fxnote[author=JA]{\textcolor{mygreen}{Optočlen}}
K uživatelskému řízení senzoru jsou zde 4 programovatelná tlačítka a tři indikační diody.

Senzor je tedy schopen zaznamenávat data ze 4 vstupů a teplotu z teplotního modulu.

\section{Uchycení}
Obal řídící desky je vytisknutý na 3D tiskárně, na přední straně je průhledné plexi sklo do kterého jsou vyvrtané díry na tlačítka.
Na boční straně krabičky jsou udělané dvě drážky na protažení stahovací zip pásky. Kabely jsou poté svedeny po konstrukci stroje až k periferiím.
Display je tak jednoduše nastavitelný, a pohodlně přizpůsobitelný.


\section{Program}
K programování využívám aplikaci Visual Studio Code s rozšířením PlatformIO, které je navržena k programování mikrokontrolérů. 
Zdrojový kód mám napsaný v jazyce C 
Program se skládá z několika smyček, které se pravidel spouštějí a vykonávají.
První a zároveň nejdůležitější smyčka je senzorová, v této smyčce se periodicky kontroluje stav vstupů a při změně se odešle událost na server.
Další smyčka zajišťuje pravidelné vykreslování dat na display a ostatní smyčky se starají o správný chod senzoru a o jeho aktualizaci.
Software také obsahuje ladící mód ve kterém si administrátor může zobrazit stav senzoru v mobilní aplikaci. 


\fxnote[author=JA]{\textcolor{mygreen}{Obrázek deksa => krabička}}

\newpage

\section{2. verze - speciální senzorika}

Po měsíci testování jsem vytvořil nový seznam, zaměřený pouze na monitoring výroby ponožek na specifických strojích.
Zařízení je díky tomu mnohem menší, levnější a softwarově rychlejší.

\begin{itemize}
    \item vstup pouze ze 2 periferií
    \item senzor s barevným displayem
    \item vstupní napětí již od 5V
    \item zredukování rozměrů
    \item jednoduše dostupný USB C kabel
    \item zachování možnosti externích senzorů
    \item dostačující dvě tlačítka
    \item možnost přímého napájení senzoru bez měniče
\end{itemize}

\subsection{Řídící deska}
Návrh druhé desky jsem se rozhodl udělat v open source aplikaci KiCad, která začíná být velmi populární. 
Tato aplikace podporuje mnoho rozšíření, která velmi zpříjemní návrh a zjednoduší přípravu podkladů.

V novém návrhu jsem se především zaměřoval na rozměr, aby byl co nejmenší.
Zařízení jsem chtěl nabídnout kompaktní ale zároveň s co nejlepší funkcionalitou.

Deska tedy ve výsledku obklopuje stejný modul ESP32 jako byl v předchozí generaci.

Deska tedy přišla o dvě uživatelská tlačítka a o jednu indikační LEDku. 
Možnost připojení externího senzoru na desce stále zůstala.

V senzoru se změnilo zapojení měniče napětí, tentokrát dokáže pracovat již od 5V, které následně mění na 3,3V.
Na bočních stranách desky vznikla také nová křidélka pro zasunutí do nového krytu.

\fxnote[author=JA]{\textcolor{mygreen}{krabička popis, uchycení}}
\fxnote[author=JA]{\textcolor{mygreen}{Obrázek deksa => krabička}}

\section{Uchycení}
Druhá verze využívá stejného principu uchycení krabičku ke stroji. 
Mění se zde však spojení krabičky se senzorovou deskou. 
Zde jsem desku navrhl tak, aby se dala jednoduše zasunout do kolejnic které jsou předtištěné v krabičce a následně zafixovat šroubkem ze zadní strany.
Tento návrh už má také vyřešené uchycení kabelů ke konstrukci krabičky pomocí \textcolor{mygreen}{DOPLNIT}.
Konstrukce je také kompletně prachotěsná.




\newpage