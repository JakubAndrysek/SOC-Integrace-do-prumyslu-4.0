\chapter{Vývoj}
Na této práci jsem začal pracovat v~únoru 2020, kdy jsem si jako úplný nováček četl dokumentaci k~jazyku PHP. 
Původní verzi webového rozhraní jsem začal navrhovat v~čistém PHP. Tento způsob byl však velmi zdlouhavý a~neefektivní.
Po měsíci práce v~čistém PHP jsem přešel na framework Nette, který mi práci zjednodušil a~posunul mě velmi rychle dál. 


%SECTION
\section{Systém Pletačka IoT verze 1.0}
Tato verze byla vydána začátkem července 2020, kdy už systém uměl pracovat s~virtuálními senzory.


\subsection{Senzory}
Souběžně s~programováním webu jsem pracoval na softwaru pro senzory.
V~této době byly senzory schopné posílat data na server, ale neměly žádný grafický výstup ani nepodporovaly interakci s~uživatelem.

\subsection{Web}
Vznikla základní kostra webu a~postupně vznikaly první stránky.
Data ze~senzorů se pouze ukládala do databáze, ale~web s~nimi zatím neuměl pracovat.
Začínal se vyvíjet systém na zpracovávání údajů ze senzorů.


% \newpage

%SECTION
\section{Systém Pletačka IoT verze 2.0}
Druhá verze přinesla velké rozšíření systému.
Tato verze je produkčně nasazena od půlky prosince 2020 a~do teď běží bez větších problémů.


\subsection{Senzory}
Propojení systému s~novou verzí senzorů, které nově podporují nahrávání aktualizací přes WiFi, mají přehlednější zobrazování dat na displej a~dokážou upozornit na výpadek sítě.



\subsection{Web}
Největší proměnou prošlo webové rozhraní. Domovská stránka má přehledné zobrazování stavů senzorů. U~senzorů se zobrazují důležitá data a~pomocí grafů se dají data jednoduše porovnávat.
Přibylo nastavování směn a~hromadné přidávání senzorů.



% %SECTION
% \section{Systém Pletačka IoT verze 3.0}
% Nadále pracuji na další verzi, která přinese nové funkcionality a~vylepší stávající. 


% \fxnote[author=JPA]{\textcolor{mygreen}{pokud chceš mít tuhle sekci, tak napiš co přesně přinese, jinak to sem neuváděj}}


\newpage