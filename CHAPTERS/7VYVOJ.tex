\chapter{Vývoj}
Na této práci jsem začal pracovat v únoru 2020, kdy jsem si jako úplný nováček četl dokumentaci k jazyku PHP. 
Původní verzi webového rozhraní jsem začal navrhovat v čistém PHP, tento způsob byl však velmi zdlouhavý a neefektivní.
Po měsíci práce v čistém PHP jsem přešel na framework Nette, který mi práci zjednodušil a posunul mě velmi rychle dál. 


%SECTION
\section{Systém Pletačka IoT verze 1.0}
Tato verze byla vydána začátkem července, kdy už systém uměl pracovat s virtuálními senzory.


\subsection{Senzory}
Souběžně s programováním webu jsem pracoval na softwaru pro senzory.
V této době byly senzory schopné posílat data na server, ale neměli žádný grafický výstup ani nepodporovaly interakci s uživatelem.

\subsection{Web}
Vznikla základní kostra webu a postupně vznikaly první stránky.
Data ze senzorů se zatím pouze ukládala do databáze a web s nimi zatím neuměl pracovat.
Začínal se vyvíjet systém na zpracovávání údajů ze senzorů.


% \newpage

%SECTION
\section{Systém Pletačka IoT verze 2.0}
Druhá verze přinesla velké rozšíření systému.
Tato verze byla vydána v půlce prosince a prošla dlouhodobým testováním.


\subsection{Senzory}
Senzory nově podporují nahrávání aktualizací přes WiFi, dále mají přehlednější zobrazování dat na displej a dokážou upozornit na výpadek sítě.
Vyšla také nová generace senzorů které jsou mnohem menší a lépe přizpůsobené výrobně ponožek.

\subsection{Web}
Největší proměnou prošlo webové rozhraní. Domovská stránka má přehledné zobrazování stavů senzorů, u senzorů se zobrazují důležitá data a pomocí grafů se dají data jednoduše porovnávat.
Přibylo také nastavování směn a hromadné přidávání senzorů.



%SECTION
\section{Systém Pletačka IoT verze 3.0}
Nadále pracuji na další verzi, která přinese nové funkcionality a vylepší stávající. 

\newpage