\chapter{Integrace do průmyslu 4.0}
Průmysl 4.0 se do České republiki dostal okolo roku 2013 a od té doby se stále více rozšiřuje v průmyslových firmách.
Jedno z klíčových míst je IoT, neboli internet věcí, který nám zajišťuje vzdálenou kontrolu a řízení strojů.
Další vlastností těchto systémů je zaznamenávání a následné ukládání dat do datových úložišť.
Moderní IoT řídící systémy se snaží proniknou co nejvíce do hloubky řídících systém a zpřesnit tak naměřená data důležitá pro optimalizaci produkce.   

\section{Popis}
Při návrhu mého systému jsem se snažil řídit se těmito zásadami a navrhnout tak co nejmodernější a provozně efektivní systém.
Základem bylo zhodnocení stávající situace a navržení možného řešení.

Jedntlivé problémy
\begin{itemize}
    \item dlouhá doba stání nečinných strojů
    \item ruční počítání vyprodukovaného zboží
    \item absence historického přehledu produkce
\end{itemize}

\fxnote[author=JA]{\textcolor{mygreen}{GRAF}}

\section{Řešení}
Mým řešením je tedy návrh moderního systému, který by celý tento provoz monitoroval a přehledně ..
Dále se také snažím o zhodnocení jednotlivých směn a jejich porovnání v grafech a naměřených číslech.
Systém je neustále vyvýjen a rozšiřován o nové funkcionality navržené firmou.


%SECTION
\section{Naszaení}
Jak jsem již psal, tento systém je aktuálně nasazen ve firmě ROTEX Vysočina s.r.o, která se věnuje výrobou ponožek. 
Firma pracuje ve dvousměnném provozu a deně se zde vyprodukuje v průměru **** párů ponožek.
Díky mému systémy by se ve firmě dala zoptimalizovat produkce a výkon strojů a zefektivnit tak následnou výrobu. 

\fxnote[author=JA]{\textcolor{mygreen}{Obrázek pletárny}}

\newpage