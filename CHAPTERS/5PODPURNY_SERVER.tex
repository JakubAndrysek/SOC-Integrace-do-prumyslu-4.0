\chapter{Podpůrný server}
Podpůrný server vznikl jako rozšíření pro senzory.
Server je naprogramovaný v Pythonu a běží na Raspberry Pi společně s webovým serverem.\newline
Zdrojový kód na Githubu: \href{https://github.com/Pletacka-IoT/Pletacka-python-server}{Pletacka-python-server}\cite{PL_PY}


%SECTION
\section{Kontrola senzorů}
Hlavním úkolem tohoto serveru je detekce zapnutých senzorů.
Na serveru běží takzvaný Watchdog, jde o hlídacího psa, který každé čtyři vteřiny čaké na zprávu ze senzoru.
Touto zprávou se senzor nahlásí, že je zapnutý, pokud takováto zpráva nedojde deset vteřin, je senzor prohlášen za vypnutý a v databázi se označí jako neaktivní.
Tato jednoduchá metoda umožňuje kontrolovat velké množství senzorů jednoduchým programem.

%SECTION
\section{Automatické aktualizace}
Bezdrátová aktualizace senzorů je nová funkcionalita kterou nadále vyvíjím a rozšiřuji.
Senzory aktuálně podporují rychlou aktualizaci přes WiFi ze vzdáleného počítače.
V počítači stačí vybrat číslo senzoru a nová verze programu se pomocí WiFi připojení nahraje do senzoru.

V nové verzi přibude také hromadná aktualizace senzorů a systém na udržování aktuálních verzí systému ve všech senzorech.




\newpage