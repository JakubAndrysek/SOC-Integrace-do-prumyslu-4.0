\chapter{Podpůrný server}
Podpůrný server vznikl jako rozšíření pro senzory.
Server je naprogramovaný v~Pythonu a~běží na Raspberry Pi společně s~webovým serverem.\newline
Zdrojový kód na Githubu: \href{https://github.com/Pletacka-IoT/Pletacka-python-server}{Pletacka-python-server}\cite{PL_PY}


%SECTION
\section{Kontrola senzorů}
Hlavním úkolem tohoto serveru je detekce zapnutých senzorů.
Na serveru běží takzvaný Watchdog.
Jde o~periodickou smyčku, která každé čtyři~vteřiny čeká na zprávu ze senzoru.
Touto zprávou se senzor nahlásí, že je zapnutý. Pokud takováto zpráva nedojde do deseti vteřin, je senzor prohlášen za vypnutý a~v~databázi se označí jako neaktivní.


%SECTION
\section{Automatické aktualizace}
Bezdrátová aktualizace senzorů je nová funkcionalita, kterou nadále vyvíjím a~rozšiřuji.
Senzory podporují rychlou aktualizaci přes WiFi ze~vzdá\-le\-né\-ho počítače.
V~počítači~stačí vybrat číslo senzoru a~nová verze programu se pomocí WiFi připojení nahraje do senzoru.





\newpage