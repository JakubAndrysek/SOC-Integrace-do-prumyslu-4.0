\chapter{Konkurence}
Tento systém je velice specifický a nedá se srovnávat jako celek. 
Potenciální konkurenci tohoto systému jsem tedy rozdělil na dvě části.

\begin{itemize} % odrážkový seznam
    \item Hardware
    \item Software
\end{itemize}



%SECTION
\section{Hardware}




\subsection{PLC}
PLC neboli programovatelný logický automat je průmyslový počítač k řízení automatizovaných procesů.
Automaty zpracovávají data v reálném čase a s co nejkratší odezvou.
PLC jsou velmi modulární a dají se skládat různě dohromady, podle potřeby uživatele.
Je možné je také připojit do sítě a vzdálené řídit. !!!!!!Zařízení jsou dodávány s uzavřeným programovacím softwarem 



\subsection{Industruino}
Firma Industruino\cite{INDUSTRINO} se zabývá vývojem zařízení pro průmyslovou automatizaci založenou na platformě Arduino.
Zařízení splňují průmyslové standardy a jsou navržený pro montáž na  DIN lištu. Firma nabízí také moduly s WiFi nebo se SIM konektivitou.


\subsection{Hardwario}

Hardwario\cite{HARDWARIO} je česká firma, která nabízí průmyslové IoT stavebnice.
Cílem této firmy je nabídnou IoT průmyslové IoT řešení, které si sami sestavíte podle představ.
Firma se zaměřuje na nízkoenergetické moduly s vydrží několika let.
Produkt je nabízen bez softwaru, ten je nutné naprogramovat v externích aplikacích.
Nevýhodou tohoto produktu je vysoká pořizovací cena.





\subsection{Arduino}
Arduino \cite{ARDUGREENHOUSE} je otevřený (open source) projekt který se díky své ceně a jednoduchosti na používání rozšířil po celém světě.
Arduino má v nabídce přes deset různých modelů. Desky jsou velice univerzální a jsou velmi často využívány na kutilské projekty.
K Arduinu také existuje velké množství shieldů, které základním modulům dodávají novou funkcionalitu. 
Jde například o WiFi moduduly, motorové moduly, nebo různé teplotní senzory.
Desky Arduino se programují v jazyce Wiring\footnote{Jazyk vytvořený k programování mikročipů}, nebo v jazyce C++. 

\fxnote[author=JA]{\textcolor{mygreen}{Obrázek Arduina}}


\subsection{Srovnání}

První tří zmíněné platformy jsou hojně využívány v průmyslu a řídí většinu automatizovaných procesů, jejich nasazení je složité a celé systémy jsou velmi drahé.

\begin{enumerate}
    \item Jednoduché uchycení
    \item Průmyslové napětí 5-25 V
    \item Open source
    \item Barevný display
    \item Bezdrátová konektivita ve výchozím provedení
		\item Moderní konektor USB C
  \end{enumerate}


	\begin{table}[]
		\centering
		\begin{tabular}{|l|l|l|l|l|l|l|}
			\hline
													& 1 & 2 & 3 & 4 & 5 & 6 \\ \hline
			PLC                  & \cmark & \cmark & \xmark & \cmark & \cmark & \xmark \\ \hline
			Industruino          & \cmark & \cmark & \cmark & \xmark & \cmark & \xmark \\ \hline
			Hardwario            & \cmark & \xmark & \cmark & \xmark & \cmark & \xmark \\ \hline
			Arduino              & \xmark & \xmark & \cmark & \xmark & \xmark & \xmark \\ \hline
			\textbf{Moje řešení} & \cmark & \cmark & \cmark & \cmark & \cmark & \cmark \\ \hline
		\end{tabular}
	\end{table}
	

\newpage

%SECTION
\section{Software}


\subsection{NodeRED}
NodeRED je jednoduché grafické prostředí k programování IoT zařízení. 
Hlavní výhodou této aplikace je, že celá běží jako webová stránka. 
Tím umožňuje uživateli rychlou práci bez nutnosti instalovat speciální aplikace.
NodeRED programování stojí na principu propojování jednotlivých uzlů.
Pro složitější projekty může být složité nastavit propojení bloků
Ve složitějších projektech mohou být bloky dosti nepřehledné a složité na úpravu.


!!Open source


\subsection{Blynk}
Blynk je platforma pro vzdálené ovládání IoT projektů.
Základem platformy je jednoduchá mobilní aplikace pro nastavování a vyčítání dat.
Aplikace nabízí velké množství widgetů které se připínají na zobrazovací panel.
Na osobní projekty do pěti zařízení je aplikace zdarma, jinak je nutné platit měsíční poplatky.


\subsection{Home asistent}




\subsection{Porovnání}
Lorem ipsum dolor sit amet, consectetur adipiscing elit.
Aliquam nunc magna, sollicitudin id leo eu, viverra congue risus.
Aliquam consequat ipsum ut erat placerat consequat nec at diam. 
Aenean est odio, molestie sit amet nunc in, pretium luctus elit. 
Donec imperdiet orci vel porttitor placerat. 








\subsection{Univerzálnost}
Lorem ipsum dolor sit amet, consectetur adipiscing elit.
Aliquam nunc magna, sollicitudin id leo eu, viverra congue risus.
Aliquam consequat ipsum ut erat placerat consequat nec at diam. 
Aenean est odio, molestie sit amet nunc in, pretium luctus elit. 
Donec imperdiet orci vel porttitor placerat. 
Proin ut hendrerit elit, ultricies accumsan urna. 
Vivamus condimentum lorem viverra lectus finibus, nec volutpat turpis auctor.
Cras quis felis non lorem consectetur interdum eu eu sem. 
Proin sit amet feugiat metus. 
Ut vitae orci a enim vestibulum porta. 
\newpage