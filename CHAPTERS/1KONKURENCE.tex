\chapter{Konkurence}
Tento systém je velice specifický a~nedá se srovnávat jako celek. 
Potenciální konkurenci tohoto systému jsem tedy rozdělil na dva celky.

\begin{itemize} % odrážkový seznam
    \item Hardware
    \item Software
\end{itemize}



%SECTION
\section{Hardware}

\fxnote[author=JA]{\textcolor{mygreen}{Přidat obrázky}}

\subsection{PLC}
PLC neboli programovatelný logický automat je průmyslový počítač k~řízení automatizovaných procesů.
Automaty zpracovávají data v~reálném čase a~s~co nejkratší odezvou.
PLC jsou velmi modulární a~dají se skládat různě dohromady, podle potřeby uživatele.


% \fxnote[author=JPA]{Industruino bych nahradil bych Controllinem: https://www.controllino.com/product/controllino-maxi/}

\subsection{Controllino}
Firma Controllino\cite{CONTROLLINO} se zabývá vývojem zařízení pro průmyslovou automatizaci založenou na platformě Arduino.
Zařízení nabízí několik vstupních a~výstupních pinů, pomocí kterých si uživatel může připojit své senzory a~následně automatizovat některé procesy. 


% \subsection{Industruino}
% Firma Industruino\cite{INDUSTRINO} se zabývá vývojem zařízení pro průmyslovou automatizaci založenou na platformě Arduino.
% Zařízení splňují průmyslové standardy a~jsou navržená pro montáž na  DIN lištu. Firma nabízí také moduly s~WiFi nebo se SIM konektivitou.


\subsection{Hardwario}
Hardwario\cite{HARDWARIO} je česká firma, která nabízí průmyslové IoT stavebnice.
Cílem této firmy je nabídnou průmyslové IoT řešení, které si sami sestavíte podle svých představ.
Firma se zaměřuje na nízkoenergetické moduly s~vydrží několika let.
% Nevýhodou tohoto produktu je jeho vysoká pořizovací cena. 



\subsection{Arduino}
Arduino \cite{ARDUINO} je otevřený (open source) projekt který se díky své nízké ceně a~jednoduchosti na používání rozšířil po celém světě.
Arduino má v~nabídce přes deset různých modelů. Desky jsou univerzální a~jsou velmi často využívány na kutilské projekty.
K~Arduinu také existuje velké množství shieldů, které základním modulům dodávají další funkcionalitu. 
% Jde například o~WiFi moduduly, motorové moduly, nebo různé teplotní senzory.
Desky Arduino se programují v~jazyce Wiring, vytvořeném přímo pro programování mikrokontrolérů, nebo v~jazyce C++. 




\subsection{Srovnání}

První tří zmíněné platformy jsou hojně využívány v~průmyslu a~řídí většinu automatizovaných procesů, jejich nasazení je složité a~celé systémy jsou velmi drahé.\newline
Požadavky na platformu
\begin{enumerate}
    \item Připraveno na montáž na zařízení
    \item Průmyslové napětí 5-25 V
    \item Open source
    \item Barevný displej
    \item Bezdrátová konektivita ve výchozím provedení
		\item Moderní konektor USB-C
  \end{enumerate}


	\begin{table}[]
		\centering
		\begin{tabular}{|l|l|l|l|l|l|l|}
			\hline
			\B{Hardware}		& 1 	 & 2 	  & 3 	   & 4 		& 5 	 & 6 	  \\ \hline
			PLC                 & \cmark & \cmark & \xmark & \xmark & \xmark & \xmark \\ \hline
			Controllino         & \cmark & \cmark & \cmark & \xmark & \xmark & \xmark \\ \hline
			Hardwario           & \cmark & \xmark & \cmark & \xmark & \cmark & \xmark \\ \hline
			Arduino             & \xmark & \xmark & \cmark & \xmark & \xmark & \xmark \\ \hline
			\B{Moje řešení} 	& \cmark & \cmark & \cmark & \cmark & \cmark & \cmark \\ \hline
		\end{tabular}
		\caption{Tabulka srovnání hardwarové konkurence.}
		\label{tab:COMPARATION}
	\end{table}
	

\newpage

%SECTION
\section{Software}


\subsection{Node-RED}
Node-RED je jednoduché grafické prostředí k~programování IoT zařízení. 
Hlavní výhodou této aplikace je, že celá běží jako webová stránka. 
Tím umožňuje uživateli rychlou práci bez nutnosti instalovat speciální aplikace.
Node-RED programování stojí na principu propojování jednotlivých uzlů.
% Pro složitější projekty může být složité nastavit propojení bloků.
Ve složitějších projektech mohou být ovšem bloky dosti nepřehledné a~složité na úpravu.


\subsection{Blynk}
Blynk je platforma pro vzdálené ovládání IoT projektů.
Základem platformy je jednoduchá mobilní aplikace pro nastavování a~vyčítání dat.
Aplikace nabízí velké množství widgetů které se připínají na zobrazovací panel.
Na osobní projekty do pěti zařízení je aplikace zdarma, jinak je nutné platit měsíční poplatky.


\subsection{Home Assistant}
Home Assistant je software pro řízení chytrých domácností. 
Systém dokáže pracovat s~více než 1700 službami.
Připojená zařízení se konfigurují pomocí textového souboru.
Aplikace také dokáže integrovat mnoho rozšíření, například ESPHome.
To slouží k~ovládání mikrokontrolérů ESP které jsou hojně rozšířené v~kutilské komunitě.
Aplikace také nabízí přehledné widgety k~rychlému zobrazení nejdůležitějších dat. 


\subsection{Porovnání}
Node-RED a~Home Assistant jsou projekty s~otevřeným zdrojovým kódem, utvářené komunitou, díky tomu jsou tyto systémy velmi modulární a~rychle se rozvíjejí. % přizpůsobují novým zařízením.
Naopak Blynk je uzavřená platforma zaměřená na firmy a~vývojáře.
Můj systém spojuje užitečné vlastnosti ze všech těchto systémů a~nabízí je jako celek v~podobě systému Pletačka IoT.
% Všechny ty tyto platformy pracují s~velkým množstvím různých programovatelných IoT senzorů.