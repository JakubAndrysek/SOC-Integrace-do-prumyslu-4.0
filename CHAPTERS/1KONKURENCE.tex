\chapter{Konkurence}
Tento systém je velice specifický a nedá se srovnávat jako celek. 
Potenciální konkurenci tohoto systému jsem tedy rozdělil na dvě části.

\begin{itemize} % odrážkový seznam
    \item Hardware
    \item Software
\end{itemize}



%SECTION
\section{Hardware}




\subsection{PLC}
Lorem ipsum dolor sit amet, consectetur adipiscing elit.
Aliquam nunc magna, sollicitudin id leo eu, viverra congue risus.
Aliquam consequat ipsum ut erat placerat consequat nec at diam. 
Aenean est odio, molestie sit amet nunc in, pretium luctus elit. 
Donec imperdiet orci vel porttitor placerat. 
Proin ut hendrerit elit, ultricies accumsan urna. 
Vivamus condimentum lorem viverra lectus finibus, nec volutpat turpis auctor.
Cras quis felis non lorem consectetur interdum eu eu sem. 
Proin sit amet feugiat metus. 
Ut vitae orci a enim vestibulum porta. 


\subsection{Industruino}
Firma Industruino\cite{INDUSTRINO} se zabývá vývojem zařízení pro průmyslovou automatizaci založenou na platformě Arduino.
Zařízení splňují průmyslové standardy a jsou navržený pro montáž na  DIN lištu. Firma nabízí také moduly s WiFi nebo se SIM konektivitou.


\subsection{Hardwario}

Hardwario\cite{HARDWARIO} je česká firma, která nabízí průmyslové IoT stavebnice.
Cílem této firmy je nabídnou IoT průmyslové IoT řešení, které si sami sestavíte podle představ.
Firma se zaměřuje na nízkoenergetické moduly s vydrží několika let.
Produkt je nabízen bez softwaru, ten je nutné naprogramovat v externích aplikacích.
Nevýhodou tohoto produktu je vysoká pořizovací cena.





\subsection{Arduino}
Arduino \cite{ARDUGREENHOUSE} je otevřený (open source) projekt který se díky své ceně a jednoduchosti na používání rozšířil po celém světě.
Arduino má v nabídce přes deset různých modelů. Desky jsou velice univerzální a jsou velmi často využívány na kutilské projekty.
K Arduinu také existuje velké množství shieldů, které základním modulům dodávají novou funkcionalitu. 
Jde například o WiFi moduduly, motorové moduly, nebo různé teplotní senzory.
Desky Arduino se programují v jazyce Wiring\footnote{Jazyk vytvořený k programování mikročipů}, nebo v jazyce C++. 

\fxnote[author=JA]{\textcolor{mygreen}{Obrázek Arduina}}


\subsection{Srovnání}



%SECTION
\section{Software}

Open source, Jednoduché uchycení, průmyslové napájení,Barevný display, bezdrátová konektivita, průmyslový standart


\subsection{Univerzálnost}
Lorem ipsum dolor sit amet, consectetur adipiscing elit.
Aliquam nunc magna, sollicitudin id leo eu, viverra congue risus.
Aliquam consequat ipsum ut erat placerat consequat nec at diam. 
Aenean est odio, molestie sit amet nunc in, pretium luctus elit. 
Donec imperdiet orci vel porttitor placerat. 
Proin ut hendrerit elit, ultricies accumsan urna. 
Vivamus condimentum lorem viverra lectus finibus, nec volutpat turpis auctor.
Cras quis felis non lorem consectetur interdum eu eu sem. 
Proin sit amet feugiat metus. 
Ut vitae orci a enim vestibulum porta. 
\newpage