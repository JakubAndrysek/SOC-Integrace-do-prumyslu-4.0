\chapter{Princip fungování Pletačka IoT}
V předchozích kapitolách byly popsány jednotlivé část systému Pletačka IoT.
V této kapitole bude celý systém popsán jako celek.


\subsection{Sběr dat}
První a tou nejdůležitější částí je získávání dat pomocí senzorů.
Jakmile senzor zaznamená jakoukoliv změnu, okamžitě tuto zprávu odesílá na server.
Odesílání probíhá skrze senzorové API, kde se nejdříve senzor ověří a následně se stav zapíše do databáze k příslušnému senzoru.
Po zapsání do databáze se vrátí do senzoru zpráva o provedení zápisu. 


\subsection{Vyhodnocování dat}
Dalším krokem je zpracovávání surových dat z databáze.
K tomuto účelu běží na serveru výběrové API, které je automaticky spouštěné v nastavený čas.
Jde o generování širších výběrů dat, hodinové, denní, měsíční a roční výběry.
Tyto výběry se následně ukládají do databáze k danému senzoru.
Generování těchto dat probíhá převážně v noci, kdy je server nejméně vytížen.


\subsection{Zobrazování dat}
Posledním krokem je zobrazení dat uživateli.
Je to jediná část se kterou se běžný uživatel dostane do kontaktu.
Proto je nutné aby zobrazení bylo co nejrychlejší a pro uživatele co nejpříjemnější.
K rychlému zobrazování se využívají předgenerované výběry, ke kterým se rychle dopočítají nově nasbíraná data.

\fxnote[author=JA]{\textcolor{mygreen}{schéma sběr - vyhodnocení - zobrazení}}

\subsection{Konektivita}
Webové stránky se dají jednoduše zobrazit na počítači či notebooku.
Stránky jsou také responzivní a správně se zobrazují i na mobilních zařízeních.
Přístup k webu je pouze z vnitřní sítě firmy, to zajišťuje dostatečnou bezpečnost pro celý systém.


\newpage