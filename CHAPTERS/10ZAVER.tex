\chapter*{Závěr}

Cílem této práce bylo navrhnout ucelený systém, který dokáže:

\begin{itemize}
    \item počítat upletené ponožky
    \item zjišťovat poruchovost strojů
    \item porovnávat jednotlivé pracovní směny
    \item monitorovat průběh výroby
\end{itemize}

Všechny tyto vytyčené cíle se mi podařilo splnit. Systém nadále běží ve firmě ROTEX Vysočina s.r.o \cite{ROTEX} a~pomáhá v~běžném provozu.
Můj systém se stal nedílnou součástí výrobního procesu a~analyzuje průběh výroby.

Systém mám k~1. únoru 2021 nasazen na deseti pletacích strojích a~po dobu provozu zaznamenal již přes padesát tisíc upletených ponožek.
Celý systém je nasazený krátkou dobu, abych dokázal porovnat produktivitu před nasazením tohoto systému s~daty, po nasazení.

Velkým přínosem pro firmu je porovnávání pracovních směn, díky kterým zaměstnavatel ihned vidí rozdíly mezi produktivitou práce v~daném čase.

Díky SOČ jsem se naučil navrhovat plošné spoje, rozšířil jsem si obzory v~elektronice a při vývoji jsem si vyzkoušel práci s měřícími přístroji. 
Také jsem se naučil programovat v~jazyce PHP a~vytvářet komplexní webové systémy.

V~budoucnu bych chtěl tento systém rozšířit na všechny pletací stroje a~pokrýt tak celou výrobnu.
Taktéž pokračuji na vylepšování webové aplikace a~plánuji ji rozšířil o~další funkce.
Jde například o~export dat do tabulek.

Tuto práci můžete najít na adrese: \url{https://github.com/JakubAndrysek/SOC-Integrace-do-prumyslu-4.0/blob/master/text.pdf}.

Všechny zdrojové kódy a DPS k projektu jsou k dispozici na \url{https://github.com/Pletacka-IoT} pod MIT licencí.


\newpage