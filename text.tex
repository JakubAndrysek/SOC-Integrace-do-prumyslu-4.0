\documentclass{template/socthesis}

\usepackage{subcaption} 
\usepackage{amsmath} 
\usepackage{enumitem} 
\usepackage{hyperref} % reference
\usepackage{gensymb} % balíček symbolů
\usepackage{booktabs}

\usepackage[toc,page]{appendix}
\usepackage{color} % balíček pro obarvování textů
\usepackage{xcolor}  % zapne možnost používání barev, mj. pro \definecolor
\definecolor{mygreen}{RGB}{0,150,0} % nastavení barev odkazů 
\usepackage{listings} % balíček pro formátování zdrojových kódů 
\usepackage[author=,status=final]{fixme} % vkládání poznámek  
% dva módy (status): draft (poznámky se zobrazují v PDF) / final (poznámky se nezobrazují v PDF)
\usepackage{multirow}

\lstset { %
    language=C++,
    backgroundcolor=\color{black!5}, % set backgroundcolor
    basicstyle=\footnotesize,% basic font setting
}

\addbibresource{text.bib} % soubor s bibliografií
\nocite{*}

\titlecz{Integrace do průmyslu 4.0} % český název práce
\titleen{Integration into industry 4.0} % anglický název práce
\author{Jakub Andrýsek} % jméno a příjmení autora
\field{18} % obor (pouze číslo, zbytek vysází šablona - číslo oboru viz http://www.soc.cz/obory-soc/)
\school{Gymnázium Brno, Vídeňská, příspěvková organizace} % celý název školy
\mentor{Mgr. Jaroslav Páral} % jméno a příjmení školitele
\mentorstatement{Mgr. Jaroslava Párala} % jméno a příjmení ve druhém pádě 

% Změňte, pokud se liší
%\region{Jihomoravský} % kraj
\placefooter{Brno 2021} % místo a rok

% hinty k používání balíčků hyperref, url, hyperlink a hypertarget
% \usepackage{hyperref} % balíček pro hypertextové odkazy
% \url{www.odkaz.cz}
% \href{http://www.odkaz.cz}{Text který bude jako odkaz}
% \hyperlink{label}{proklikávací_text} - odkaz na text 
% \hypertarget{label}{cíl_odkazu} - cíl odkazu 

\begin{document} % konec preambule dokumentu

\maketitle % vysází titulky

\makecopyrightstatement{V~Brně} % místo

% poděkování
\makethanks{Děkuji svému školiteli Mgr. Jaroslavovi Páralovi za obětavou pomoc, podnětné připomínky a~hlavně nekonečnou trpělivost, kterou mi během práce poskytoval.}

\pagestyle{empty}

\section*{Anotace}
\color{mygreen}
Anotace má za úkol stručně popsat cíle práce a velmi stručný úvod k tématu. 
Většinou bývá použit první odstavec, nebo jiná část úvodu.
\color{black}

Zahradničení je dnes naprosto běžnou zájmovou činností. Mnoho lidí mající takovou zálibu je ovšem velmi časově vytížených. Kromě práce se musí starat mnohdy i o~rodinu a~na péči o~rostliny jim často jednoduše nezbývá čas. Jedním z~těchto lidí je i můj táta, který mě inspiroval k~vytvoření PROTOPlantu -- systému pro snadnou a~levnou automatizaci skleníku. 

Cílem práce je vytvořit univerzální a~dostupný systém pro automatizaci skleníku, který by usnadnil péči o~rostliny časově vytíženým lidem. 

\subsection*{Klíčová slova}
\color{mygreen}
Klíčová slova.
Snažte se najít alespoň 5, ideálně i více klíčových slov, která jednoduše vystihují vaši práci.
\color{black}

automatizace skleníku, ESP32, PROTOPlant, automatizace, open-source hardware, open-source software

\newpage % pokud se anotace vleze na jednu stránku (což by měla), tento rádek zakomentuj

\vspace{20mm}

\section*{Annotation}
\color{mygreen}
Zde přijde anglický překlad anotace.
\color{black}

Gardening is a~very common hobby today. However, many people who likes this activity doesn't have enough time for it. 
Beside work, they have to take care of their families and after this, they don't have any time to take care of plants. 
My dad is exactly this kind of man. 
And that inspired me to create PROTOPlant -- system for easy and cheap greenhouse automation.

Goal of this thesis is to create universal and available system for greenhouse automation, that will make it easier for these people to take care of their plants.

\subsection*{Keywords}
\color{mygreen}
Klíčová slova - jejich překlad do angličtiny.
\color{black}

greenhouse automation, ESP32, PROTOPlant, automation, open-source hardware, open-source software

\newpage
\pagestyle{plain}

\tableofcontents % vysází obsah

%%% Začátek práce
\setcounter{figure}{0}
\setcounter{table}{0}
\newpage

% zde můžeš s pomocí příkazu \input{cesta k souboru} vložit soubory; doporučuji každou větší kapitolu dát do samostatného souboru pro větší přehlednost

% Úvod práce
\input{CHAPTERS/UVOD.tex}

% Motivace
\input{CHAPTERS/KAPITOLA1.tex}

% Konkurence
\input{CHAPTERS/KAPITOLA2.tex}

% Zaver prace
\input{CHAPTERS/ZAVER.tex}
\newpage

\appendix
\addcontentsline{toc}{chapter}{Přílohy}

% Prilohy
\input{CHAPTERS/PRILOHY.tex}

\printbibliography[title=Literatura]
\addcontentsline{toc}{chapter}{Literatura}

\listoffigures
\addcontentsline{toc}{section}{Seznam obrázků}

\listoftables
\addcontentsline{toc}{section}{Seznam tabulek}

\end{document}
